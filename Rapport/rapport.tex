\documentclass[11pt]{article}

\usepackage[utf8]{inputenc}
\usepackage[T1]{fontenc}
\usepackage[francais]{babel}
\usepackage{fancyhdr}
\usepackage[scale=0.7]{geometry}
\usepackage{siunitx}
\usepackage{amsmath}
\usepackage{graphicx}
\usepackage{hyperref}

\pagestyle{fancy}
\renewcommand{\footrulewidth}{1pt}
\fancyfoot[C]{\textbf{page \thepage}}
\fancyfoot[L]{Polytech Nantes}
\fancyfoot[R]{2016 -- 2017}
\fancyhead[L]{Équipe 13}
\fancyhead[R]{INFO4}

\title{Mini-projet de C++ : Google Hashcode 2016\\Rapport de l'équipe 13}
\author{Vincent \bsc{Cotineau} \and Anthony \bsc{Griffon} \and Benjamin \bsc{Landry} \and Hugo \bsc{Pigeon} \and Pierre \bsc{Pétillon}}

\makeatletter
\let\ps@plain\ps@fancy
\makeatother

\begin{document}

	\maketitle
	
	% ajouter une introduction

	\section{Algorithme naïf}
	
		\subsection{Principe}
		
			Le but de cette solution est d'obtenir un premier résultat à partir d'un principe relativement simple. Nous avons effectué quelques optimisations pour minimiser le temps de calcul, mais ce n'était pas l'objectif prioritaire de cette solution.
			
			Le principe est donc le suivant : chaque satellite prend la photo la plus proche qu'il peut atteindre. Ainsi, on applique le traitement sur chaque satellite.
		
		\subsection{Optimisation}

\end{document}
